\documentclass{rapportHenallux}
\usepackage{lipsum}
\usepackage{listings}
\usepackage{lettrine}
\usepackage{tabto}
\usepackage{wrapfig}
\usepackage{multirow}
\addbibresource{biblio.bib}
\definecolor{backcolour}{rgb}{0.95,0.95,0.92}
\newcommand{\mycomment}[1]{}

\lstset{
    basicstyle=\footnotesize,
    numbers=left,
    numberstyle=\normalsize,
    numbersep=7pt,
    frame=l,
    commentstyle=\color{teal},
    keywordstyle=\color{blue},
    backgroundcolor=\color{backcolour},
    showspaces=false,
    breakatwhitespace=false,
    keepspaces=true,
    showstringspaces=false,
}
\mycomment{
\lstset{
  backgroundcolor=\color{backcolour}, commentstyle=\color{teal},
  keywordstyle=\color{magenta},
  numberstyle=\tiny\color{codegray},
  stringstyle=\color{codepurple},
  basicstyle=\ttfamily\footnotesize,
  breakatwhitespace=false,         
  breaklines=true,                 
  captionpos=b,                    
  keepspaces=true,                 
  numbers=left,                    
  numbersep=5pt,                  
  showspaces=false,                
  showstringspaces=false,
  showtabs=false,                  
  tabsize=2
}
}

\title{Rapport Henallux} %Titre du fichier

\begin{document}

%----------- Informations du rapport ---------

\titre{Template de rapport en \LaTeX} %Titre du fichier .pdf
\UE{Sécurité OS} %Nom de la UE
\sujet{IR Bac2 Groupe A} %Nom du sujet

\enseignant{Xxxx \textsc{Xxxx}} %Nom de l'enseignant

\eleves{Jean \textsc{Staffe} \\
		Justin \textsc{S.}
  }
%----------- Initialisation -------------------
        
\fairemarges %Afficher les marges
\fairepagedegarde %Créer la page de garde
\tabledematieres %Créer la table de matières

%------------ Corps du rapport ----------------
\setcounter{page}{1}
\section{Première section} 

\lipsum[3-4]%Effacer cette ligne et écrire le texte souhaité

\subsection{Subsection}

\lipsum[3-4] %Effacer cette ligne et écrire le texte souhaité

\section{Deuxième section}

\lipsum[3-5] %Effacer cette ligne et écrire le texte souhaité

%------------- Commandes utiles ----------------

\section{Quelques commandes}

Voici quelques commandes utiles\footnote{Tout est relatif !} :

%------ Pour insérer et citer une image centralisée -----

\insererfigure{images/logo_henallux.png}{3cm}{Légende de la figure}{Label de la figure}
% Le premier argument est le chemin pour la photo
% Le deuxième est la hauteur de la photo
% Le troisième la légende
% Le quatrième le label
Ici, je cite l'image \ref{fig: Label de la figure}


%------- Pour insérer et citer une équation --------------

\begin{equation} \label{eq: exemple}
\rho + \Delta = 42
\end{equation}

L'équation \ref{eq: exemple} est cité ici. 

% ------- Pour écrire des variables ----------------------

Pour écrire des variables dans le texte, il suffit\cite{einstein} de mettre le symbole \$ entre le texte souhaité comme : constante $\rho$. 
\begin{lstlisting}[language=Python, title=test.py]
import numpy as np
    
def incmatrix(genl1,genl2):
    m = len(genl1)
    n = len(genl2)
    M = None #to become the incidence matrix
    VT = np.zeros((n*m,1), int)  #dummy variable
    
    #compute the bitwise xor matrix
    M1 = bitxormatrix(genl1)
    M2 = np.triu(bitxormatrix(genl2),1) 

    for i in range(m-1):
        for j in range(i+1, m):
            [r,c] = np.where(M2 == M1[i,j])
            for k in range(len(r)):
                VT[(i)*n + r[k]] = 1;
                VT[(i)*n + c[k]] = 1;
                VT[(j)*n + r[k]] = 1;
                VT[(j)*n + c[k]] = 1;
                
                if M is None:
                    M = np.copy(VT)
                else:
                    M = np.concatenate((M, VT), 1)
                
                VT = np.zeros((n*m,1), int)
    
    return M
\end{lstlisting}

\section{Les énumérations}
\begin{enumerate}
    \item Date d'enregistrement du nom de domaine:
    \begin{itemize}
        \item 22-01-2013
    \end{itemize}
    \item Sous-domaines de megacorpone:
    \begin{itemize}
        \item admin.megacorpone.com
        \item beta.megacorpone.com \url{https://cp.megacorpone.net/}
        \item fs1.megacorpone.com \url{https://cp.megacorpone.net/}
    \end{itemize}
\end{enumerate}

\section{Les tableaux}
\begin{tabular}{|l|c|r|}
  \hline
  colonne 1 & colonne 2 & colonne 3 \\
  \hline
  1.1 & 1.2 & 1.3 \\
  2.1 & 2.2 & 2.3 \\
  \hline
\end{tabular}

\begin{center}
\begin{tabular}{||c c c c||} 
 \hline
 Col1 & Col2 & Col2 & Col3 \\ [0.5ex] 
 \hline\hline
 1 & 6 & 87837 & 787 \\ 
 \hline
 2 & 7 & 78 & 5415 \\
 \hline
 3 & 545 & 778 & 7507 \\
 \hline
 4 & 545 & 18744 & 7560 \\
 \hline
 5 & 88 & 788 & 6344 \\ [1ex] 
 \hline
\end{tabular}
\end{center}

\begin{tabular}{ |p{3cm}||p{3cm}|p{3cm}|p{3cm}|  }
 \hline
 \multicolumn{4}{|c|}{Country List} \\
 \hline
 Country Name or Area Name& ISO ALPHA 2 Code &ISO ALPHA 3 Code&ISO numeric Code\\
 \hline
 Afghanistan   & AF    &AFG&   004\\
 Aland Islands&   AX  & ALA   &248\\
 Albania &AL & ALB&  008\\
 Algeria    &DZ & DZA&  012\\
 American Samoa&   AS  & ASM&016\\
 Andorra& AD  & AND   &020\\
 Angola& AO  & AGO&024\\
 \hline
\end{tabular}

\section{Les images encadrées}
\begin{wrapfigure}{l}{0.25\textwidth}
\includegraphics[width=0.9\linewidth]{images/logo_henallux.png} 
\caption{Caption1}
\label{fig:wrapfig}
\end{wrapfigure}
Luigi Galvani (1737-1798) était un physicien, inventeur, professeur d’anatomie et médecin italien. Il est surtout connu pour ses travaux sur l’électricité animale. Pour lui, les animaux produisent eux-mêmes un courant électrique qui permet aux muscles de se contracter. Il a découvert que les muscles des grenouilles mortes se contractaient lorsqu’ils étaient touchés par des arcs métalliques. Cette découverte a conduit à de nouvelles recherches sur la relation entre l’électricité et la vie. Galvani a également mené des expériences sur l’électrophysiologie et a découvert que les nerfs et les muscles produisent de l’électricité. Ses travaux ont ouvert la voie à de nouvelles recherches dans le domaine de la neurologie.

\begin{lstlisting}[language=Java, title=test.sql]
const sql = require('mssql')
const { faker } = require('@faker-js/faker');
faker.locale = 'fr';
const config = {
 user: "sa",
 password: "p@s5w0rd",
 database: "JSMarket",
 server: 'docker.staffe.net',
 options: {
 encrypt: true,
 trustServerCertificate: true
 }
}
const numOfData = 15000;
var incr = 0;
while(incr != numOfData){
 sql.connect(config, function (err) {
 
 if (err) console.log(err);
 
 // create Request object
 var request = new sql.Request();
 
 // query to the database and get the records
 var ville = Math.floor(Math.random() * 27) + 1;
 var random = Math.floor(Math.random() * 3);
 if(random == 2){
 var valueR = Math.floor(Math.random() * 10) + 1;
 }else{
 var valueR = "NULL"
\end{lstlisting}
\newpage \thispagestyle{empty}
~~
\vspace{150px}

{\fontsize{24}{25}\selectfont \textbf{Remerciements}}
\vspace{10px}

\lettrine[findent=2pt]{\textbf{T}}{ }est lettrine 

\tabto{6cm} \lipsum[2-3]

% Bibliographie
\fairebibliographie

% Table des figures
\newpage \listoffigures

% Table des programmes

\end{document}
