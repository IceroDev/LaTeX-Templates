\documentclass{rapportHenallux}
\usepackage{lipsum}
\usepackage{listings}
\lstset{
    basicstyle=\footnotesize,
    numbers=left,
    numberstyle=\normalsize,
    numbersep=8pt,
    frame=l,
    commentstyle=\color{teal},
    keywordstyle=\color{blue},
    backgroundcolor=\color{white},
    showspaces=false,
    breakatwhitespace=false,
    keepspaces=true,
    showstringspaces=false,
}
\title{Rapport Henallux} %Titre du fichier

\begin{document}

%----------- Informations du rapport ---------

\titre{Template de rapport en \LaTeX} %Titre du fichier .pdf
\UE{Sécurité OS} %Nom de la UE
\sujet{IR Bac2 Groupe A} %Nom du sujet

\enseignant{Xxxx \textsc{Xxxx}} %Nom de l'enseignant

\eleves{Jean \textsc{Staffe} \\
		Justin \textsc{S.}
  }
%----------- Initialisation -------------------
        
\fairemarges %Afficher les marges
\fairepagedegarde %Créer la page de garde
\tabledematieres %Créer la table de matières

%------------ Corps du rapport ----------------


\section{Première section} 

\lipsum[3-4]%Effacer cette ligne et écrire le texte souhaité

\subsection{Subsection}

\lipsum[3-4] %Effacer cette ligne et écrire le texte souhaité

\section{Deuxième section}

\lipsum[3-5] %Effacer cette ligne et écrire le texte souhaité

%------------- Commandes utiles ----------------

\section{Quelques commandes}

Voici quelques commandes utiles\footnote{Tout est relatif !} :

%------ Pour insérer et citer une image centralisée -----

\insererfigure{images/logo_henallux.png}{3cm}{Légende de la figure}{Label de la figure}
% Le premier argument est le chemin pour la photo
% Le deuxième est la hauteur de la photo
% Le troisième la légende
% Le quatrième le label
Ici, je cite l'image \ref{fig: Label de la figure}


%------- Pour insérer et citer une équation --------------

\begin{equation} \label{eq: exemple}
\rho + \Delta = 42
\end{equation}

L'équation \ref{eq: exemple} est cité ici. 

% ------- Pour écrire des variables ----------------------

Pour écrire des variables dans le texte, il suffit de mettre le symbole \$ entre le texte souhaité comme : constante $\rho$. 
\begin{lstlisting}[language=Python, title=test.py]
import numpy as np
    
def incmatrix(genl1,genl2):
    m = len(genl1)
    n = len(genl2)
    M = None #to become the incidence matrix
    VT = np.zeros((n*m,1), int)  #dummy variable
    
    #compute the bitwise xor matrix
    M1 = bitxormatrix(genl1)
    M2 = np.triu(bitxormatrix(genl2),1) 

    for i in range(m-1):
        for j in range(i+1, m):
            [r,c] = np.where(M2 == M1[i,j])
            for k in range(len(r)):
                VT[(i)*n + r[k]] = 1;
                VT[(i)*n + c[k]] = 1;
                VT[(j)*n + r[k]] = 1;
                VT[(j)*n + c[k]] = 1;
                
                if M is None:
                    M = np.copy(VT)
                else:
                    M = np.concatenate((M, VT), 1)
                
                VT = np.zeros((n*m,1), int)
    
    return M
\end{lstlisting}
\end{document}
